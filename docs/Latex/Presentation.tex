\documentclass[xcolor=sgvnames,serifs,notes,compress,professionalfont]{beamer}
\usepackage{amsfonts,amsmath}
\usepackage{enumerate}
\usepackage{graphicx}
\usepackage{booktabs}
\usepackage{xcolor}
\usepackage{colortbl}
\usepackage{tikz}
\usetheme{CambridgeUS}
\usecolortheme{whale}
\setbeamercolor{frametitle}{fg=black}
\setbeamertemplate{caption}[numbered]

\newcommand\ballref[1]{%
	\tikz \node[circle, shade,ball color=structure.fg,inner sep=0pt,%
	text width=8pt,font=\tiny,align=center] {\color{white}\ref{#1}};
}

\begin{document}
	
\title{Predicting the bug fixing likelihood}
\author{Florian Spychiger}
\institute{University of Zurich}

\begin{frame}
	\titlepage
\end{frame}

\section{Introduction}
\begin{frame}
\frametitle{Roadmap}
	\begin{enumerate}
		\item Background
		\item Problem Formulation \& Goal
		\item Data
		\item Solution Approach
		\item Results
		\item Q\&A
	\end{enumerate}
\end{frame}

\section{Background}
\begin{frame}
The annual cost of software bugs is estimated at \$59.5 billion\footnote{P Bhattacharya and I Neamtiu, “Fine-­grained incremental learning and multi-­feature tossing graphs to improve bug triaging”, Software Maintenance (ICSM) 2010 (ieeexplore.ieee.org)}. For the Eclipse project, there are thousands of bugs reported. An efficent bug-triaging can help developers to focus their resources and thus, save companies a lot of money. 
\frametitle{Background Information}
\end{frame}

\section{Problem Formulation \& Goal}
\begin{frame}
\frametitle{Problem Formulation \& Goal}
\begin{alertblock}{Problem}Bug-triaging is an important, but labor-intensive process if done manually.\end{alertblock}
\kern 3em
\begin{block}{Goal}Train a bug-triaging machine, which predicts whether a bug is likely to be fixed.\end{block}
\end{frame}

\section{Data}
\begin{frame}
\frametitle{Raw Data}
The Eclipse data set can be found at
\url{https://github.com/ansymo/msr2013-bug_dataset}.

The raw data set consists of 12 tables:
\begin{table}
	\centering
	\begin{tabular}{|l|l|}
		\hline
		\multicolumn{2}{|l|}{Eclipse Bug Data Set}\\
		\hline
		reports & priority\\
		assigned\_to & product\\
		bug\_status & resolution\\
		cc\footnote{The data has been newly formated with Excel VBA.} & severity\\
		component & short\_desc\\
		op\_sys & version\\
		\hline 
	\end{tabular}
	\caption{Tables of the bug data set.}
	\label{tab:DataTables}
\end{table}	

\end{frame}

\begin{frame}
\frametitle{Data Preselection}
After a visual exploratory analysis, four datasets were excluded:

\begin{columns}[T] % align columns
	\begin{column}{.48\textwidth}
	\begin{table}
		\centering
		\begin{tabular}{|l|l|l}
			\hline
			\multicolumn{2}{|l|}{Eclipse Bug Data Set*}\\
			\hline
			reports & \cellcolor{red!25}priority \ballref{ite:first}\\
			assigned\_to & product\\
			bug\_status & resolution\\
			cc & \cellcolor{red!25}severity \ballref{ite:second}\\
			component &\cellcolor{red!25}short\_desc \ballref{ite:third}\\
			op\_sys & \cellcolor{red!25}version \ballref{ite:fourth}\\
			\hline 
		\end{tabular}
		\caption{Excluded data tables.}
		\label{tab:ExTables}
	\end{table}	
	*All duplicate bugs are excluded.
	\end{column}%
	\hfill%
	\begin{column}{.48\textwidth}
		\begin{enumerate}
			\item\label{ite:first} Priority is set by the assignee, but as we want to help them triaging the bugs, we exlude it.
			\item\label{ite:second} Severity is currently set by the triaging team.
			\item\label{ite:third} The descriptions are hard to encode.
			\item\label{ite:fourth} The version dataset is quite messy and sometimes it is not clear which version is being referred to.
		\end{enumerate}
		
	\end{column}%
\end{columns}
\end{frame}

\begin{frame}
\frametitle{Data Model}
\begin{figure}
\includegraphics[width=\textwidth]{pictures/ERDiagram/ERDiagram.pdf}    
\caption{ER model of data used.}
\label{fig:ER}
\end{figure}
\end{frame}

\section{Solution Approach}
\begin{frame}
\frametitle{Feature Creation}
From the data model, the feature matrix $X$ is constructed with:
\begin{align*}
x_1 &= \text{OpenTime (Open - Close )} && [discrete]\\
x_2 &= \text{Assignments (Nr. of assignees)}&& [discrete]\\
x_3 &= \text{CC (Nr. of interested parties)}&& [discrete]\\
x_4 &= \text{Product (Affected product)}&& [discrete]\\
x_5 &= \text{OS (Major OS)}&& [discrete]\\
x_6 &= \text{SuccessAssignee (Success rate of Assignee)}&& [proportion]\\
x_7 &= \text{SuccessReporter (Success rate of Reporter)}&& [proportion]\\
x_8 &= \text{Component (The affected subcomponent)}&& [discrete]\\
x_9 &= \text{Social (Past bug collaborations)}&& [discrete]\\
x_{10}&= \text{Equal (Reporter equals Assignee)} && [binary]
\end{align*}
The labels are $y= $ Fixed with values in $\{0,1\}$.
\end{frame}

\begin{frame}
\frametitle{Univariate Analysis}
\begin{figure}
	\includegraphics[height=0.75\textheight]{pictures/boxplots.png}    
	\caption{Boxplots of the features and the label.}
	\label{fig:boxplots}
\end{figure}

\end{frame}

\begin{frame}
\frametitle{Correlation Analysis}
\begin{figure}
	\includegraphics[height=0.75\textheight]{pictures/correlations.png}    
	\caption{Correlations of the features and the labels.}
	\label{fig:corr}
\end{figure}
\end{frame}

\begin{frame}
\frametitle{Models}
We consider 6 models:
\begin{enumerate}
	\item Naive Bayes
	\item Logistic Regression
	\item Random Forest
	\item Boosting Classifier
	\item Support Vector Machine
	\item Neural Network
\end{enumerate}
We split the data set into a training (50\%), a cross-validation (25\%) and a test (25\%) set. The training set is used to train the models and we calibrate the parametes on the cross-validation set. The final accuracy is caculated on the test set.
\end{frame}

\section{Results}
\begin{frame}
\frametitle{Accuracy}
We achieve the following accuracies on the test set:
\begin{table}
	\begin{tabular}{|l|l|}
		\hline
		Naive Bayes & 82.8098\%\\\hline
		Logistic Regression & 84.9409\%\\\hline
		\cellcolor{green!25}Random Forest & \cellcolor{green!25}86.1529\%\\\hline
		Boosting Classifier & 85.4661\%\\\hline
		Support Vector Machine &  85.9105\%\\\hline
		Neural Network\footnote{Results are not exactly reproducible, as some randomness with GPU usage cannot be avoided.} & 86.1125\%\\
		\hline 
	\end{tabular}
	\caption{Accuracies of the models.}
	\label{tab:Accs}
\end{table}
\end{frame}

\begin{frame}
\frametitle{ROC-Curves}
\begin{figure}
	\includegraphics[height=0.75\textheight]{pictures/rocs.png}    
	\caption{ROC-Curves of all models.}
	\label{fig: rocs}
\end{figure}
\end{frame}

\section{Q\&A}
\begin{frame}
\frametitle{Q\&A}
\begin{center}
	The code of the project can be found at
\end{center}
	\kern 2em
\begin{center}
	\url{https://github.com/Speaker90/BusinessAnalytics_RPIcase}
\end{center}
\end{frame}

\end{document}